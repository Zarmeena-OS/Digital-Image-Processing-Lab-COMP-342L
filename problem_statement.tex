\documentclass[11pt,a4paper]{article}
\usepackage[utf8]{inputenc}
\usepackage[T1]{fontenc}
\usepackage{geometry}
\usepackage{graphicx}
\usepackage{booktabs}
\usepackage{hyperref}
\geometry{margin=2.5cm}

\title{Problem Statement and Full Implementation Report\\
\large Digital Image Processing (COMP-342L) --- All 12 Labs}
\author{[Your Name]}
\date{\today}

\begin{document}
\maketitle

\begin{abstract}
This document states the problem addressed by the course project and summarizes the full implementation across all 12 labs. The project focuses on \textbf{low-light and uneven illumination image enhancement} using classical methods (histogram-based, Retinex-based, gamma/log transforms, denoising, and color preservation). Labs 3--12 implement a complete pipeline from contrast baseline to evaluation and paper draft.
\end{abstract}

\tableofcontents
\newpage

%=============================================================================
\section{Problem Statement}
%=============================================================================

\subsection{Topic}
Low-light and uneven illumination image enhancement using classical and (optionally) hybrid pipelines.

\subsection{Objective}
Design, implement, and evaluate a practical image enhancement pipeline for low-light and unevenly lit images. Compare classical methods in terms of:
\begin{itemize}
  \item Contrast improvement
  \item Noise behavior
  \item Color fidelity
\end{itemize}
Establish a baseline and optional extensions (e.g., denoising + enhancement, simple learning-based module) so that the full pipeline can be written up as a short research paper by Lab 12.

\subsection{Deliverables}
\begin{itemize}
  \item Code and figures for Labs 3--11.
  \item Short paper (4--6 pages): introduction, related work, method, experiments, discussion, conclusion (Lab 12).
\end{itemize}

%=============================================================================
\section{Repository and Environment}
%=============================================================================

\subsection{Structure}
\begin{itemize}
  \item \texttt{Lab 1/}--\texttt{Lab 12/}: Per-lab code and README.
  \item \texttt{requirements.txt} (Python 3.10+), \texttt{requirements-py39.txt} (Python 3.9).
  \item \texttt{problem\_statement.tex}: This document.
\end{itemize}

\subsection{Setup}
\begin{verbatim}
python -m venv venv
source venv/bin/activate   # or venv\Scripts\activate on Windows
pip install -r requirements.txt
\end{verbatim}

%=============================================================================
\section{Lab-by-Lab Implementation (All 12 Labs)}
%=============================================================================

\subsection{Lab 1}
Placeholder; no implementation required for the research pipeline.

\subsection{Lab 2 --- Image I/O and RGB Channels}
\textbf{Purpose:} Load images and display RGB channels.\\
\textbf{Files:} \texttt{Lab 2/Ali Hamza's Lab/Part 1.py}, \texttt{Part 2.py}; Zarmeena's Lab equivalents.\\
\textbf{Run:} \texttt{cd "Lab 2/Ali Hamza's Lab"; python Part 1.py} (or \texttt{Part 2.py}).

\subsection{Lab 3 --- Contrast Baseline (Global HE vs CLAHE)}
\textbf{Purpose:} First block of the enhancement pipeline; compare global histogram equalization with Contrast Limited Adaptive Histogram Equalization (CLAHE) on grayscale images.\\
\textbf{Files:} \texttt{Lab 3/clahe\_vs\_he.py}, \texttt{Lab 3/Content/}.\\
\textbf{Output:} \texttt{Lab3\_comparison.png} (original | global HE | CLAHE).\\
\textbf{Run:} \texttt{cd "Lab 3"; python clahe\_vs\_he.py}.\\
\textbf{Research angle:} Global HE often over-enhances noise; CLAHE limits contrast gain and preserves local detail (cite F1000Research 2021, MDPI Algorithms 2024).

\subsection{Lab 4 --- Gamma and Log Transforms}
\textbf{Purpose:} Power-law (gamma) and log transforms for tone mapping; compare with Lab 3.\\
\textbf{Files:} \texttt{Lab 4/gamma\_log\_transforms.py}.\\
\textbf{Output:} \texttt{Lab4\_gamma\_log.png} (original | gamma=0.5 | gamma=1.5 | log).\\
\textbf{Run:} \texttt{cd "Lab 4"; python gamma\_log\_transforms.py}.

\subsection{Lab 5 --- Single-Scale Retinex (SSR)}
\textbf{Purpose:} Illumination--reflectance decomposition: $R = \log(I) - \log(G*I)$ with Gaussian $G$.\\
\textbf{Files:} \texttt{Lab 5/ssr.py}.\\
\textbf{Output:} \texttt{Lab5\_ssr.png} (original | SSR $\sigma=15$ | SSR $\sigma=80$).\\
\textbf{Run:} \texttt{cd "Lab 5"; python ssr.py}.

\subsection{Lab 6 --- Multi-Scale Retinex (MSR)}
\textbf{Purpose:} MSR = weighted sum of SSR at multiple scales ($\sigma = 15, 80, 250$).\\
\textbf{Files:} \texttt{Lab 6/msr.py}.\\
\textbf{Output:} \texttt{Lab6\_msr.png} (original | MSR).\\
\textbf{Run:} \texttt{cd "Lab 6"; python msr.py}.

\subsection{Lab 7 --- Color Preservation (CLAHE on V in HSV)}
\textbf{Purpose:} Apply CLAHE only to the V (value) channel in HSV; H and S unchanged to avoid color shift.\\
\textbf{Files:} \texttt{Lab 7/color\_preservation.py}.\\
\textbf{Output:} \texttt{Lab7\_color\_preserved.png} (original RGB | enhanced, color preserved).\\
\textbf{Run:} \texttt{cd "Lab 7"; python color\_preservation.py}.

\subsection{Lab 8 --- Denoising + Enhancement Pipeline}
\textbf{Purpose:} Bilateral filter (edge-preserving denoising) followed by CLAHE.\\
\textbf{Files:} \texttt{Lab 8/denoise\_enhance.py}.\\
\textbf{Output:} \texttt{Lab8\_denoise\_enhance.png} (original | bilateral | bilateral+CLAHE).\\
\textbf{Run:} \texttt{cd "Lab 8"; python denoise\_enhance.py}.

\subsection{Lab 9 --- Objective Metrics (PSNR, SSIM)}
\textbf{Purpose:} Compute PSNR and SSIM between reference and enhanced images.\\
\textbf{Files:} \texttt{Lab 9/metrics.py}.\\
\textbf{Output:} \texttt{Lab9\_metrics.png} and console PSNR/SSIM.\\
\textbf{Run:} \texttt{cd "Lab 9"; python metrics.py}.

\subsection{Lab 10 --- Batch Evaluation on Dataset}
\textbf{Purpose:} Run CLAHE on all images in \texttt{Content/} (and fallback dirs); save to \texttt{outputs/}.\\
\textbf{Files:} \texttt{Lab 10/batch\_eval.py}, \texttt{Lab 10/Content/}.\\
\textbf{Output:} \texttt{Lab 10/outputs/<name>\_clahe.png} per input.\\
\textbf{Run:} \texttt{cd "Lab 10"; python batch\_eval.py}.

\subsection{Lab 11 --- Ablation / Parameter Study}
\textbf{Purpose:} Vary CLAHE \texttt{clip\_limit} and \texttt{kernel\_size}; report PSNR and SSIM vs reference.\\
\textbf{Files:} \texttt{Lab 11/ablation.py}.\\
\textbf{Output:} Console table; \texttt{Lab11\_ablation.txt}.\\
\textbf{Run:} \texttt{cd "Lab 11"; python ablation.py}.

\subsection{Lab 12 --- Paper Draft}
\textbf{Purpose:} Short paper: introduction, related work, method (pipeline), experiments, discussion, conclusion.\\
\textbf{Files:} \texttt{Lab 12/paper\_draft.md}, \texttt{Lab 12/README.md}.\\
No script; edit \texttt{paper\_draft.md} or build this LaTeX document.

%=============================================================================
\section{Pipeline Summary}
%=============================================================================

The research pipeline (Labs 3--12) implements:
\begin{enumerate}
  \item Contrast baseline (HE, CLAHE)
  \item Gamma/log transforms
  \item Single-scale and multi-scale Retinex
  \item Color preservation (HSV)
  \item Denoising + enhancement (bilateral + CLAHE)
  \item Metrics (PSNR, SSIM)
  \item Batch evaluation
  \item Ablation study
  \item Paper draft
\end{enumerate}

%=============================================================================
\section{References}
%=============================================================================

\begin{itemize}
  \item F1000Research (2021). Enhancement of digitized X-ray films using Contrast-Limited Adaptive Histogram Equalization (CLAHE).
  \item MDPI Algorithms (2024). Impact of image enhancement using CLAHE on spine X-ray segmentation with U-Net, Mask R-CNN, and transfer learning.
  \item He et al., Single image haze removal using dark channel prior (TPAMI 2010) --- related dehazing.
\end{itemize}

\end{document}
